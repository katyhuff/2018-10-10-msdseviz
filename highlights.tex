\documentclass{article}

\usepackage[acronym,toc]{glossaries}
\include{acros}

\makeglossaries


\usepackage{lineno}
\usepackage{xspace}

%%%% packages and definitions (optional)
\usepackage{placeins}
\usepackage{booktabs} % nice rules (thick lines) for tables
\usepackage{microtype} % improves typography for PDF
\usepackage{hhline}
\usepackage{amsmath}

%\usepackage[demo]{graphicx}
%\usepackage{caption}
%\usepackage{subcaption}

\usepackage{booktabs}
\usepackage{threeparttable, tablefootnote}

\usepackage{tabularx}
\newcolumntype{b}{>{\hsize=1.0\hsize}X}
\newcolumntype{s}{>{\hsize=.5\hsize}X}
\newcolumntype{m}{>{\hsize=.75\hsize}X}
\newcolumntype{x}{>{\hsize=.25\hsize}X}


% tikz %
\usepackage{tikz}
\usetikzlibrary{positioning, arrows, decorations, shapes}

\usetikzlibrary{shapes.geometric,arrows}
\tikzstyle{process} = [rectangle, rounded corners, minimum width=3cm, minimum height=1cm,text centered, draw=black, fill=blue!30]
\tikzstyle{object} = [ellipse, rounded corners, minimum width=3cm, minimum height=1cm,text centered, draw=black, fill=green!30]
\tikzstyle{arrow} = [thick,->,>=stealth]

% Cyclus
\newcommand{\Cyclus}{\textsc{Cyclus}\xspace}%

% hyperref %
\usepackage[hidelinks]{hyperref}

\begin{document}
\title{Workshopping Figures and Visualizations}
\author{Kathryn D. Huff$^{1}$\\
        $^{1}$Dept. of Nuclear, Plasma, and Radiological Engineering, \\
University of Illinois at Urbana-Champaign, \\
Urbana, IL}

\date{}
\maketitle 
These plots are the result of hard work from three PhD students: Jin Whan Bae, Gwendolyn 
Chee, and Andrei Rykhlevskii.

\section*{Recycling Fuel in a Molten Salt Breeder Reactor \cite{rykhlevskii_modeling_2018}}

\textbf{PhD student Andrei Rykhlevskii ran long time scale depletion (neutron 
transport) and online reprocessing (fancy chemistry) calculations to determine 
the equilibrium composition of the nuclear fuel and other characteristics in a 
particular molten salt reactor design from the 70s.} 

Figure~\ref{fig:breeding_den} shows the neutron capture reaction rate
distribution for $^{232}$Th normalized by the total neutron flux for initial
and equilibrium states. The distribution reflects the spatial distribution of
$^{233}$Th production in the core. The thorium-232 then $\beta$-decays to
$^{233}$Pa, which is the precursor for $^{233}$U production. Accordingly, this
characteristic represents the breeding distribution in the \gls{MSBR} core.
Spectral shift does not cause significant changes in power nor in breeding
distribution. Even after 20 years of operation, most of the power is still
generated in zone I and the majority of $^{233}$Th is
produced in zone II.

\begin{figure}[ht!] % replace 't' with 'b' to force it to \centering
  \includegraphics[width=\textwidth]{power_distribution_eq.png}
  \caption{Normalized power density for equilibrium fuel salt
  composition.}
  \label{fig:pow_den}
\end{figure}
\begin{figure}[ht!] % replace 't' with 'b' to force it to \centering
  \includegraphics[width=\textwidth]{breeding_distribution_eq.png}
  \caption{$^{232}$Th neutron capture reaction rate normalized by total flux
  for equilibrium fuel salt composition.}
  \label{fig:breeding_den}
\end{figure}

In a \gls{MSBR} reprocessing scheme, the only external feed material flow  is
$^{232}$Th. Figure~\ref{fig:th_refill} shows the $^{232}$Th feed rate
calculated for 60 years of reactor operation. The $^{232}$Th feed rate
fluctuates significantly as a result of the batch-wise nature of this online
reprocessing approach. For example, the large spikes of up to 36 kg/day in a
thorium consumption occurs every 3435 days. This is required due to batch-wise
removal of strong absorbers (Rb, Sr, Cs, Ba). The corresponding effective
multiplication factor increase (Figure~\ref{fig:keff}) and breeding
intensification leads to additional $^{232}$Th consumption.

\begin{figure}[ht!] % replace 't' with 'b' to force it to \centering
  \includegraphics[width=\textwidth]{Th_refill_rate.png} \caption{$^{232}$Th
  feed rate over 60 years of \gls{MSBR} operation.}
  \label{fig:th_refill}
\end{figure}

\FloatBarrier
\section*{Benchmarking : Agent-Based Simulation more Informative than Spreadsheets \cite{bae_standardized_2018}}
\textbf{MS student Jin Whan Bae compared an agent based fuel cycle simulator 
        with estimates from a spreadsheet (via DOE). The rich, discrete 
        facility and discrete material information available from the agent 
based simulation captures dynamics that the benchmark was forced to approximate 
at the fleet level.}

A code-to-code benchmarking exercise postulated a nuclear fuel cycle transition 
from current reactors to new ones. 
Figure \ref{fig:waiting_monthly} shows the amount of cooled \gls{UNF} waiting 
for
reprocessing. The value is calculated by subtracting the cumulative difference 
between
the cooled inventory and the \gls{UNF} reprocessing throughput.
The oscillation is between the cooled inventory in the storage facility before 
(high)
and after (low) it sends its inventory for reprocessing.

Figure \ref{fig:rep} shows the reprocessing throughput, which oscillates around
the benchmark solution. No oscillation exists in the beginning because the
\gls{LWR} \gls{UNF} reprocessing plant throughput peaks at 2,000 tons per year.


\begin{figure}[htbp!]
    \begin{center}
            \includegraphics[width=\textwidth]{./fuel_discharge_monthly.png}
    \end{center}
        \caption{Inventory of discharged \gls{UNF} in mandatory cooling storage.}
    \label{fig:fuel_discharge_monthly}
\end{figure}


\begin{figure}[htbp!]
    \begin{center}
            \includegraphics[width=\textwidth]{./waiting_monthly.png}
    \end{center}
        \caption{Inventory of discharged and cooled \gls{UNF} waiting for reprocessing.}
    \label{fig:waiting_monthly}
\end{figure}


\begin{figure}[htbp!]
    \begin{center}
            \includegraphics[width=\textwidth]{./rep.png}
    \end{center}
        \caption{Annual reprocessing throughputs.}
    \label{fig:rep}
\end{figure}

\FloatBarrier
\section*{Validating a Predictive Simulator by Predicting the Past \cite{chee_validation_2018}}

\textbf{PhD student Gwendolyn Chee sought to validate our predictive nuclear 
fuel cycle simulator by predicting the past. That is, she set up a simulation 
with initial conditions reflecting our US nuclear reactor fleet and used the 
Cyclus simulator to determine the cumulative isotopics of US spent fuel over 
the last many decades. She compared this to a database holding the DOE's best 
guess (using high fidelity calculations of each reactor) at the same data.}

Figures \ref{fig:absolute_diff_all_51} and \ref{fig:absolute_diff_all_33} show
the cumulative nuclear spent fuel isotopic mass difference between \gls{UNFSTANDARDS} 
\gls{UDB} and simulated \Cyclus data
in 5 year intervals for burnup of 51 GWD/MTU and 33 GWD/MTU correspondingly.
The \Cyclus data that assumed 33 GWD/MTU burnup deviated less than the
\Cyclus data that assumed 51 GWD/MTU burnup. This is apparent for $^{236}$U,
$^{242}$Pu and $^{240}$Pu. They are similar for the isotopes on the left side
of both figures. With an exception of $^{239}$Pu having a substantially larger
difference for 33 GWD/MTU than 51 GWD/MTU.

\begin{figure}[htb] % replace 't' with 'b' to force it to be on the bottom
    \centering
        \includegraphics[height=0.30\textheight]{absolute_diff_all_51}
        \caption{The absolute difference between spent fuel mass calculated by 
        \gls{UNFSTANDARDS} \gls{UDB} and \Cyclus for each isotope for 51 GWD/MTU burnup. Positive difference indicates \Cyclus mass estimate is larger.}
    \label{fig:absolute_diff_all_51}
\end{figure}

\begin{figure}[htb] % replace 't' with 'b' to force it to be on the bottom
        \includegraphics[height=0.30\textheight]{absolute_diff_all_33}
        \caption{The absolute difference between spent fuel mass calculated by \gls{UNFSTANDARDS} \gls{UDB} and \Cyclus for each isotope for 33 GWD/MTU burnup. Positive difference indicates \Cyclus mass estimate is larger.}
        \label{fig:absolute}
\end{figure}



\FloatBarrier
%%%%%%%%%%%%%%%%%%%%%%%%%%%%%%%%%%%%%%%%%%%%%%%%%%%%%%%%%%%%%%%%%%%%%%%%%%%%%%%%
\bibliographystyle{plain}
\bibliography{2018-huff-msdseviz}
\end{document}
